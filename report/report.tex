\documentclass[12pt,a4paper]{report}

\usepackage[spanish,es-nodecimaldot]{babel}
\usepackage[utf8]{inputenc}
\usepackage[T1]{fontenc}
\usepackage{csquotes}
\usepackage{lmodern}
\usepackage{amsmath,amssymb,amsfonts}
\usepackage{graphicx}
\usepackage{booktabs}
\usepackage{multirow}
\usepackage{url}
\usepackage{hyperref}
\usepackage{enumitem}
\usepackage{geometry}

\geometry{margin=2.5cm}

\title{Agente autónomo de ciberseguridad basado en aprendizaje por refuerzo}
\author{Javier Rivero Iglesias}
\date{}

\begin{document}

\maketitle

\tableofcontents

\chapter{Introducción}

En este Trabajo Fin de Grado se explora el uso de técnicas de aprendizaje por refuerzo
(\emph{Reinforcement Learning}, RL) para el diseño de agentes defensivos en ciberseguridad.
El objetivo global es desarrollar un agente capaz de tomar decisiones de defensa
de forma autónoma y adaptativa frente a tráfico malicioso, partiendo inicialmente
de datos etiquetados y, en fases posteriores, de entornos de red más dinámicos.

En esta primera versión de la memoria se recoge, sobre todo, la fase experimental
centrada en la clasificación de ataques utilizando el conjunto de datos NSL-KDD,
comparando un agente RL con modelos supervisados clásicos.

\chapter{Definición del problema}

El problema que se aborda se puede dividir en dos fases complementarias:

\begin{enumerate}[label=\textbf{F\arabic*.}, leftmargin=*]
    \item \textbf{Fase 1: clasificación de tráfico malicioso}.\\
    A partir de un conjunto de datos etiquetado de tráfico de red (NSL-KDD),
    se plantea la detección binaria \emph{normal} / \emph{ataque}
    formulada como un problema de decisión para un agente RL.
    En esta fase se estudia hasta qué punto un agente entrenado por refuerzo
    puede aproximarse (o no) al rendimiento de modelos supervisados como
    Random Forest.

    \item \textbf{Fase 2: defensa en un entorno de red simulado}.\\
    Una vez validada la viabilidad del enfoque y de la infraestructura
    de entrenamiento, el objetivo es extender el agente al control de
    mecanismos de defensa (por ejemplo, reglas de cortafuegos, políticas
    de bloqueo o limitación de conexiones) en un entorno de red emulado,
    donde las decisiones tienen un efecto secuencial en el tiempo.
\end{enumerate}

En este documento se desarrolla principalmente la Fase 1, dejando la Fase 2
como línea de trabajo en curso.

\chapter{Trabajos relacionados}

En esta sección se revisarán trabajos previos sobre:
\begin{itemize}
    \item Sistemas de detección de intrusos (IDS) basados en aprendizaje supervisado.
    \item Aplicación de Deep Reinforcement Learning a la detección de intrusos
          y a la defensa autónoma en redes.
    \item Conjuntos de datos de referencia en ciberseguridad (NSL-KDD, UNSW-NB15,
          CICIDS2017, etc.).
\end{itemize}

\chapter{Clasificación de ataques con aprendizaje por refuerzo}

\section{Descripción del conjunto de datos NSL-KDD}

Para la fase de clasificación se ha utilizado el conjunto de datos NSL-KDD,
que es una revisión del dataset KDD'99 clásico, eliminando redundancias y
problemas de desbalance extremo presentes en la versión original.
NSL-KDD incluye registros de conexiones de red etiquetados como \emph{normal}
o distintos tipos de ataque (DoS, probe, R2L, U2R, etc.), junto con
características estadísticas de cada flujo.

En este trabajo se considera una versión binaria del problema, en la que
se agrupan todas las variantes de ataque en una única clase \emph{malicious},
mientras que las conexiones etiquetadas como normales se mantienen
en la clase \emph{benign}. La motivación es simplificar inicialmente
la tarea a una decisión de tipo \emph{permitir/bloquear} que resulte
natural para un agente de defensa.

\subsection{Partición de entrenamiento y prueba}

El preprocesado se realiza en dos etapas:

\begin{enumerate}[label=\arabic*)]
    \item Se combinan los ficheros de entrenamiento (\texttt{KDDTrain+.TXT})
          y prueba (\texttt{KDDTest+.TXT}) originales de NSL-KDD.
    \item Se mantiene la partición que propone el propio dataset:
          \begin{itemize}
              \item Conjunto de entrenamiento: aproximadamente 125\,973 instancias.
              \item Conjunto de prueba: aproximadamente 22\,544 instancias.
          \end{itemize}
\end{enumerate}

Durante la fase de iteración rápida se utiliza también la variante reducida
\texttt{KDDTrain+\_20Percent.TXT} como conjunto de entrenamiento, manteniendo
el conjunto de prueba completo. Esta configuración reduce el coste de cómputo
permitiendo explorar distintas funciones de recompensa y arquitecturas
del agente RL.

\section{Preprocesado y representación de las observaciones}

Cada muestra del dataset incluye características numéricas y categóricas
(\texttt{protocol\_type}, \texttt{service}, \texttt{flag}, etc.).
El preprocesado implementado realiza:

\begin{itemize}
    \item Codificación \emph{one-hot} de las variables categóricas.
    \item Conversión de la etiqueta textual original (\texttt{normal} frente a tipos
          de ataque) a una etiqueta binaria:
          \[
              y =
              \begin{cases}
                  0, & \text{si la conexión es normal (benigna)},\\
                  1, & \text{si la conexión corresponde a cualquier tipo de ataque}.
              \end{cases}
          \]
    \item Normalización de las características numéricas mediante escalado
          estándar cuando se utilizan modelos supervisados clásicos.
\end{itemize}

Tras este preprocesado se obtiene una matriz de características
\(\mathbf{X} \in \mathbb{R}^{N \times d}\) y un vector de etiquetas
\(\mathbf{y} \in \{0,1\}^N\), donde \(N\) es el número de muestras y \(d\)
el número de características tras la codificación.

\section{Formulación RL del problema de clasificación}

En lugar de entrenar un modelo supervisado que aproxime directamente
la probabilidad \(\mathbb{P}(y \mid \mathbf{x})\), se plantea el problema
como una tarea de decisión para un agente de aprendizaje por refuerzo
que interactúa con un entorno de tipo Gym.

\subsection{Definición del entorno \texttt{RLDatasetDefenderEnv}}

Se define un entorno \texttt{RLDatasetDefenderEnv}, implementado en Python
sobre Gymnasium, donde cada episodio corresponde a un recorrido secuencial
por una permutación de las muestras del conjunto de entrenamiento.

En este entorno:

\begin{itemize}
    \item \textbf{Estado (observación)}: el vector de características
          \(\mathbf{x}_i \in \mathbb{R}^d\) de la muestra \(i\)-ésima.
    \item \textbf{Acciones}: el agente dispone de dos acciones discretas:
          \begin{align*}
              a = 0 &: \text{ PERMIT (permitir el tráfico)} \\
              a = 1 &: \text{ BLOCK (bloquear el tráfico)}.
          \end{align*}
    \item \textbf{Etiqueta real}: la etiqueta binaria asociada a la muestra
          se denota \(y_i \in \{0,1\}\), donde \(0\) representa tráfico benigno
          y \(1\) tráfico malicioso.
\end{itemize}

\subsection{Función de recompensa sensible al coste}

La clave del enfoque RL es definir una función de recompensa que refleje
el coste operativo de las decisiones del agente. En este trabajo se emplea
una función de recompensa binaria sensible al coste de los falsos negativos
y falsos positivos, configurada mediante un diccionario de parámetros:

\begin{center}
\begin{tabular}{ll}
\toprule
Parámetro & Significado \\
\midrule
\texttt{tp}        & Recompensa por bloquear correctamente un ataque (TP). \\
\texttt{fp}        & Penalización por bloquear tráfico benigno (FP). \\
\texttt{fn}        & Penalización por permitir un ataque (FN). \\
\texttt{omission}  & Recompensa parcial por permitir tráfico benigno. \\
\bottomrule
\end{tabular}
\end{center}

En la versión actual se ha trabajado, entre otras, con la siguiente configuración:

\begin{equation}
\text{reward\_config} =
\{ \texttt{tp}=1.0,\; \texttt{fp}=-1.0,\; \texttt{fn}=-5.0,\; \texttt{omission}=0.5 \}.
\end{equation}

Bajo esta configuración, la recompensa instantánea \(r_i\) para la muestra \(i\)
se define como:

\[
r_i =
\begin{cases}
\texttt{tp}, & \text{si } y_i = 1 \text{ y } a_i = 1 \quad (\text{ataque bloqueado}) \\
\texttt{fn}, & \text{si } y_i = 1 \text{ y } a_i = 0 \quad (\text{ataque permitido}) \\
\texttt{omission}, & \text{si } y_i = 0 \text{ y } a_i = 0 \quad (\text{benigno permitido}) \\
\texttt{fp}, & \text{si } y_i = 0 \text{ y } a_i = 1 \quad (\text{benigno bloqueado}).
\end{cases}
\]

Es decir, bloquear un ataque se recompensa con \(+1.0\), permitirlo se castiga
severamente con \(-5.0\), bloquear tráfico benigno con \(-1.0\) y permitir
tráfico benigno se considera una \emph{omisión acertada} con una recompensa
parcial de \(+0.5\).

Esta formulación permite explorar explícitamente el compromiso entre la
reducción de falsos negativos (seguridad) y la minimización de falsos
positivos (disponibilidad del servicio).

\section{Agente DQN y modelo Random Forest de referencia}

\subsection{Agente DQN}

El agente de aprendizaje por refuerzo se implementa mediante un algoritmo
\emph{Deep Q-Network} (DQN), usando la librería Stable-Baselines3.
Se emplea una política de tipo \texttt{MlpPolicy}, basada en una red
neuronal totalmente conectada con varias capas ocultas y función de
activación ReLU.

Los hiperparámetros principales utilizados en los experimentos son:

\begin{itemize}
    \item Tasa de aprendizaje: \(10^{-3}\).
    \item Tamaño del \emph{replay buffer}: 100\,000 transiciones.
    \item Tamaño de lote: 64.
    \item Factor de descuento \(\gamma = 0.99\).
    \item Frecuencia de entrenamiento: cada 4 pasos de interacción.
    \item Actualización de la red objetivo cada 10\,000 pasos.
\end{itemize}

El entrenamiento se realiza durante entre 200\,000 y 1\,000\,000 pasos
de interacción, dependiendo del experimento, y se registra tanto la
evolución de la recompensa promedio como las métricas de clasificación
sobre el conjunto de prueba.

\subsection{Modelo Random Forest de referencia}

Como referencia supervisada se entrena un clasificador Random Forest
sobre el mismo preprocesado del conjunto NSL-KDD. El modelo se entrena
con un número moderado de árboles (por ejemplo, 200 estimadores) y sin
optimización exhaustiva de hiperparámetros, ya que el objetivo principal
es disponer de un punto de comparación razonable frente al agente RL.

El Random Forest opera directamente en modo supervisado, minimizando
el error de clasificación sobre el conjunto de entrenamiento, mientras
que el agente DQN aprende a partir de señales de recompensa derivadas
de las decisiones PERMIT/BLOCK.

\section{Resultados experimentales}

En esta sección se resumen los resultados obtenidos para diferentes
configuraciones del agente DQN y del modelo Random Forest.
Cada experimento se identifica con un código (\texttt{E01}, \texttt{E02}, \dots)
y se documenta en detalle en el fichero \texttt{experiments/nslkdd\_experiments.md}
del repositorio.

\subsection{Métricas de evaluación}

Las métricas que se emplean para evaluar el rendimiento son las habituales
en problemas de detección de intrusos:

\begin{itemize}
    \item \textbf{Accuracy} global.
    \item \textbf{Precision}, \textbf{recall} y \textbf{F1-score} para cada clase.
    \item Matriz de confusión, con especial atención a:
          \begin{itemize}
              \item Falsos negativos (ataques permitidos).
              \item Falsos positivos (tráfico benigno bloqueado).
          \end{itemize}
    \item Tasa de falsos positivos (FP rate), definida como:
          \[
              \text{FP rate} = \frac{\text{FP}}{\text{TN} + \text{FP}}.
          \]
\end{itemize}

\subsection{Comparación DQN vs Random Forest (resumen cualitativo)}

De forma resumida, los resultados obtenidos muestran que:

\begin{itemize}
    \item El modelo Random Forest alcanza una \emph{accuracy} en torno al 77\%
          y un \emph{recall} de la clase de ataque cercano al 0.61, con una
          tasa de falsos positivos baja (del orden del 2--3\% del tráfico benigno).
    \item El agente DQN, incluso tras incrementar el castigo por falsos negativos
          y ajustar la recompensa por omisión, obtiene \emph{accuracies} en el rango
          71--76\%, con un \emph{recall} de ataques inferior al del Random Forest
          y una tasa de falsos positivos ligeramente inferior.
    \item En otras palabras, el agente RL tiende a aprender políticas muy
          conservadoras respecto a los falsos positivos (bloquea muy poco
          tráfico benigno), pero a costa de permitir un porcentaje relativamente
          elevado de ataques, lo que limita su utilidad como IDS principal.
\end{itemize}

Estos resultados son coherentes con la naturaleza del problema en esta fase:
se trata de una clasificación estática sobre un conjunto tabular etiquetado,
donde los modelos supervisados clásicos están muy bien adaptados. El interés
del enfoque RL aparece cuando se pasa a un entorno dinámico en el que las
decisiones del agente tienen efectos a más largo plazo, como se plantea en
la fase siguiente del TFG.

% --------------------------------------------------
% 5. Conclusiones (esqueleto)
% --------------------------------------------------
\chapter{Conclusiones y líneas futuras}

En esta primera versión de la memoria se han presentado los fundamentos
del TFG y la fase inicial de experimentación basada en NSL-KDD.
Como líneas futuras inmediatas destacan:

\begin{itemize}
    \item Completar la comparación cuantitativa entre distintas funciones
          de recompensa para el agente DQN, documentando el compromiso
          entre falsos positivos y falsos negativos.
    \item Extender el enfoque a un entorno de red simulado (por ejemplo,
          basado en Mininet o máquinas virtuales en red) donde el agente
          pueda actuar sobre reglas de defensa y observar consecuencias
          secuenciales de sus decisiones.
    \item Evaluar estrategias de aprendizaje por refuerzo profundo más
          avanzadas (PPO, A2C, etc.) y su integración con arquitecturas
          específicas para tráfico de red.
\end{itemize}

\bibliographystyle{plain}
\bibliography{bibliografia} % Ajustar nombre del .bib cuando lo tengas

\end{document}
